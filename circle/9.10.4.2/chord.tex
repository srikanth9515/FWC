\documentclass[10pt]{article}
       \usepackage[latin1]{inputenc}
       \usepackage{fullpage}
       \usepackage{color}
       \usepackage{array}
       \usepackage{longtable}
       \usepackage{calc}
       \usepackage{multirow}
       \usepackage{hhline}
       \usepackage{ifthen}
\usepackage{graphicx}
\def\inputGnumericTable{}
\usepackage[none]{hyphenat}
\usepackage{graphicx}
\usepackage{listings}
\usepackage[english]{babel}
\usepackage{graphicx}
\usepackage{caption} 
\usepackage{booktabs}
\usepackage{gensymb}
\usepackage{array}
\usepackage{amssymb} % for \because
\usepackage{amsmath}   % for having text in math mode
\usepackage{extarrows} % for Row operations arrows
\usepackage{listings}
\lstset{
  frame=single,
  breaklines=true
}
\usepackage{hyperref}
%Following 2 lines were added to remove the blank page at the beginning
\usepackage{atbegshi}% http://ctan.org/pkg/atbegshi
\AtBeginDocument{\AtBeginShipoutNext{\AtBeginShipoutDiscard}}
%New macro definitions
\newcommand{\mydet}[1]{\ensuremath{\begin{vmatrix}#1\end{vmatrix}}}
\providecommand{\brak}[1]{\ensuremath{\left(#1\right)}}
\providecommand{\norm}[1]{\left\lVert#1\right\rVert}
\newcommand{\solution}{\noindent \textbf{Solution: }}
\newcommand{\myvec}[1]{\ensuremath{\begin{pmatrix}#1\end{pmatrix}}}
\providecommand{\abs}[1]{\left\vert#1\right\vert}
\let\vec\mathbf
\begin{document}
\begin{center}
\title{\textbf{Properties of Circle}}
\date{\vspace{-5ex}} %Not to print date automatically
\maketitle
\end{center}
\setcounter{page}{1}
\section{9$^{th}$ Maths - Chapter 10}
       This is Problem-2 from Exercise 10.4
\begin{enumerate}
\item If two equal chords of a circle intersect within the circle, prove that the segments of one chord are equal to corresponding segments of other chord.

\solution:
\begin{figure}[h!]
	\begin{center} 
	  \includegraphics[width=\columnwidth]{/home/srikanth/circle/9.10.4.2/figs/c.png}
	\end{center}
\caption{Two equal chords intersecting in a circle}
\label{fig:Fig1}
\end{figure} 
Consider the circle of radius $1$ and length of chord be $1.5$
\begin{table}[h!]
	\input{/home/srikanth/circle/9.10.4.2/table/table.tex}
\caption{}
\label{table}
\end{table}
\section*{\large Construction}
Consider
\begin{eqnarray}
\vec{P}&=\myvec{\cos \theta_1\\\sin \theta_1}
\\
\vec{Q}&=\myvec{\cos \theta_2\\\sin \theta_2}
\\
\vec{R}&=\myvec{\cos \theta_3\\\sin \theta_3}
\\
\vec{S}&=\myvec{\cos \theta_4\\\sin \theta_4}
\end{eqnarray}\label{table1}
So, here
\begin{align}
	\vec{P}-\vec{Q}&=\myvec{\cos\theta_1-\cos\theta_2\\{sin \theta_1}-{\sin \theta_2}}
	\\
\norm{\vec{P}-\vec{Q}}^2&=d^2
\end{align}
		\begin{align}
	&\implies\brak{\cos \theta_1-\cos \theta_2}^2+\brak{\sin \theta_1-\sin \theta_2}^2=d^2\\
	&\implies2-2\brak{\cos\brak{\theta_1-\theta_2}}=d^2\\
	&\implies2\brak{2\brak{{\sin^2}\brak{\frac{\theta_1-\theta_2}{2}}}}=d^2\\
	&\implies\sin^2\brak{\frac{\theta_1-\theta_2}{2}}=\frac{d^2}{4}\\
	&\implies\sin\brak{\frac{\theta_1-\theta_2}{2}}=\frac{d}{2}\\
&\implies\brak{\frac{\theta_1-\theta_2}{2}}=\sin^{-1}\brak{\frac{d}{2}}\label{12}
	\end{align}
	Substituting the values from table $\eqref{table}$ in $\eqref{12}$ So we get
	\begin{align}
	\brak{\theta_1-\theta_2}&=97.1806\degree
	\end{align}
Simillary we can say that
\begin{align}
	\brak{\theta_3-\theta_4}&=97.1806\degree 		       
\end{align}
Obtain the point of intersection of line $PQ,RS$
\begin{align}
\vec{{n}_1^{\top}}\brak{\vec{X}-\vec{P}}&=0\\
\vec{{n}_2^{\top}}\brak{\vec{X}-\vec{R}}&=0\\
\vec{Omat}&=\myvec{0&1\\-1&0}\\
\vec{n}_1&=\myvec{0&1\\-1&0}\myvec{\cos \theta_1-\cos \theta_2\\\sin \theta_1-\sin \theta_2}\\
&=\myvec{\sin \theta_1-\sin \theta_2\\\cos \theta_2-\cos \theta_1}\\
\vec{n}_2&=\myvec{0&1\\-1&0}\myvec{\cos\theta_3-\cos \theta_4\\\sin \theta_3-\sin \theta_4}\\
&=\myvec{\sin \theta_3-\sin \theta_4\\\cos \theta_4-\cos \theta_3}
\end{align}
From the Result of python code by substituting $\theta_1,\theta_2,\theta_3,\theta_4$ in point $\vec{T}$ then we get the point of intersection as :
\begin{multline}
\myvec{\sin \theta_2-\sin \theta_1&\cos \theta_2-\cos \theta_1\\\sin \theta_4-\sin \theta_3&\cos \theta_4-\cos \theta_3}\vec{X}=\myvec{\cos \theta_1\brak{\sin \theta_2-\sin \theta_1}-\sin \theta_1\brak{\cos \theta_2-\cos \theta_1}\\\cos \theta_3\brak{\sin \theta_4-\sin \theta_3}-\sin \theta_3\brak{\cos \theta_4-\cos \theta_2}}\\
\end{multline}
\begin{multline}
\vec{T}=\myvec{\frac{\cos \theta_4-\cos \theta_2\brak{\sin \theta2 \cos \theta_1-\sin \theta_1 \cos \theta_2}}{\brak{\cos\theta_4-\cos \theta_2}\brak{\sin \theta_2-\sin \theta_1}-\brak{\cos \theta_2-\cos \theta_1}\brak{\sin \theta_4-\sin \theta_3}}\\[8pt]\sin \theta_1+\frac{\sin \theta_2-\sin \theta_1}{\cos \theta_2-\cos \theta_1}\brak{\frac{\cos \theta_4-\cos \theta_2\brak{\sin \theta2\cos \theta_1-\sin \theta_1\cos \theta_2}}{\brak{\cos \theta_4-\cos \theta_2}\brak{\sin \theta_2-\sin \theta_1}-\brak{\cos \theta_2-\cos \theta_1}\brak{\sin \theta_4-\sin \theta_3}}}-\cos \theta_1}
\end{multline} 	
\begin{align}
=\myvec{0.68341409\\-0.04288508}
\end{align}
From the Result of python code , we get
\begin{align}
\norm{\vec{P}-\vec{T}}&=0.5727\\
\norm{\vec{R}-\vec{T}}&=0.9272\\
\norm{\vec{Q}-\vec{T}}&=0.9272\\
\norm{\vec{S}-\vec{T}}&=0.5727
\end{align}
Hence, we proved that
\begin{align}
\norm{\vec{P}-\vec{T}}&=\norm{\vec{S}-\vec{T}}\\
\norm{\vec{Q}-\vec{T}}&=\norm{\vec{R}-\vec{T}}
\end{align} 
\end{enumerate}
\end{document}
