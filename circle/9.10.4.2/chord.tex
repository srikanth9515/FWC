\documentclass[jornel,10pt,twocolumn]{article}
%\documentclass[12pt%
			  %,landscape%
%                    ]{report}
       \usepackage[latin1]{inputenc}
       \usepackage{fullpage}
       \usepackage{color}
       \usepackage{array}
       \usepackage{longtable}
       \usepackage{calc}
       \usepackage{multirow}
       \usepackage{hhline}
       \usepackage{ifthen}
\usepackage{graphicx}
\def\inputGnumericTable{}
\usepackage[none]{hyphenat}
\usepackage{graphicx}
\usepackage{listings}
\usepackage[english]{babel}
\usepackage{graphicx}
\usepackage{caption} 
\usepackage{booktabs}
\usepackage{gensymb}
\usepackage{array}
\usepackage{amssymb} % for \because
\usepackage{amsmath}   % for having text in math mode
\usepackage{extarrows} % for Row operations arrows
\usepackage{listings}
\lstset{
  frame=single,
  breaklines=true
}
\usepackage{hyperref}
  
%Following 2 lines were added to remove the blank page at the beginning
\usepackage{atbegshi}% http://ctan.org/pkg/atbegshi
\AtBeginDocument{\AtBeginShipoutNext{\AtBeginShipoutDiscard}}


%New macro definitions
\newcommand{\mydet}[1]{\ensuremath{\begin{vmatrix}#1\end{vmatrix}}}
\providecommand{\brak}[1]{\ensuremath{\left(#1\right)}}
\providecommand{\norm}[1]{\left\lVert#1\right\rVert}
\newcommand{\solution}{\noindent \textbf{Solution: }}
\newcommand{\myvec}[1]{\ensuremath{\begin{pmatrix}#1\end{pmatrix}}}
\providecommand{\abs}[1]{\left\vert#1\right\vert}
\let\vec\mathbf
\begin{document}
\begin{center}
\title{\textbf{Properties of Circle}}
\date{\vspace{-5ex}} %Not to print date automatically
\maketitle
\end{center}

\setcounter{page}{1}

\section{9$^{th}$ Maths - Chapter 10}

This is Problem-2 from Exercise 10.4

\begin{enumerate}
\item If two equal chords of a circle intersect within the circle, prove that the segments of one chord are equal to corresponding segments of other chord.
\end{enumerate}
\solution
Consider the Circle of radius $1$ and length of chord be $1.5$
%\begin{center}
\begin{table}[ht!]
	\input{/home/srikanth/circle/9.10.4.2/table/table.tex}
\caption{}
\label{table}	
\end{table}

So, here
\begin{align}
	(\vec{P}-\vec{Q})&=\myvec{{\cos\theta_1}-{\cos\theta_2}\\{sin\theta_1}-{\sin\theta_2}}\\
	\norm{\vec{P}-\vec{Q}}^2&=d^2
		\end{align}
		\begin{align*}
	&\implies\myvec{({\cos\theta_1}-{\cos\theta_2)}^2}+\myvec{({\sin\theta_1}-{\sin\theta_2})^2}=d^2\\
	&\implies2-2\myvec{\cos(\theta_1-\theta_2)}=d^2\\
	&\implies2(2\myvec{{\sin^2}(\frac{{\theta_1}-{\theta_2}}{2})}=d^2\\
	&\implies\myvec{{\sin^2}\frac{{\theta_1}-{\theta_2}}{2}}=\frac{d^2}{4}\\
	&\implies\myvec{\sin\frac{{\theta_1}-{\theta_2}}{2}}=\frac{d}{2}\\
&\implies\myvec{\frac{{\theta_1}-{\theta_2}}{2}}=\myvec{{\sin^{-1}}(0.75)}\\
	\end{align*}
\begin{align}
	\myvec{{\theta_1}-{\theta_2}}&=97.1806 
	\end{align}
Simillary we can say that
\begin{align}
	\myvec{{\theta_3}-{\theta_4}}&=97.1806   		       
\end{align}	
Let
\begin{align}
{\theta_1=30\degree},{\theta_3=60\degree}
\end{align}
Here
\begin{align}
\vec{P}&=\myvec{\frac{\sqrt{3}}{2}\\[2pt]\frac{1}{2}}\\
\vec{Q}&=\myvec{0.3878\\-0.9217}\\
\vec{R}&=\myvec{\frac{1}{2}\\[2pt]\frac{\sqrt{3}}{2}}\\
\vec{S}&=\myvec{0.7967\\-0.6043}
\end{align}
Obtain the point of intersection of line $\vec{PQ}$,$\vec{RS}$
\begin{align}
\vec{n_1^{\top}}(\vec{T}-\vec{P})&=0\label{15}\\
\vec{n_2^{\top}}(\vec{T}-\vec{R})&=0\label{16}\\
\vec{O}&=\myvec{0&1\\-1&0}\\
\vec{n_1}&=\vec{O}(\vec{P}-\vec{Q})\\
&=\myvec{1.4217\\-0.4782}\\
\vec{n_2}&=\vec{O}(\vec{R}-\vec{S})\\
&=\myvec{1.4703\\0.2967}
\end{align}
From the Result of python code , we got the point of intersection of \eqref{15} \eqref{16}  is given as :
\begin{align}
\vec{T}&=\myvec{0.68341409\\-0.04288508}
\end{align}
From the Result of python code , we get
\begin{align}
\norm{\vec{P}-\vec{T}}&=0.5727\\
\norm{\vec{R}-\vec{T}}&=0.9272\\
\norm{\vec{Q}-\vec{T}}&=0.9272\\
\norm{\vec{S}-\vec{T}}&=0.5727
\end{align}
Hence, we proved that
\begin{align}
\norm{\vec{P}-\vec{T}}&=\norm{\vec{S}-\vec{T}}\\
\norm{\vec{Q}-\vec{T}}&=\norm{\vec{R}-\vec{T}}
\end{align} 
\begin{figure}[!h]
	\begin{center} 
	  \includegraphics[width=\columnwidth]{/home/srikanth/circle/9.10.4.2/figs/c.png}
	\end{center}
\caption{Two equal chords intersecting in a circle}
\label{fig:Fig1}
\end{figure}
\end{document}
