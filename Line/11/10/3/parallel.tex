\documentclass[12pt]{article}
\usepackage{graphicx}
%\documentclass[journal,12pt,twocolumn]{IEEEtran}
\usepackage[none]{hyphenat}
\usepackage{graphicx}
\usepackage{listings}
\usepackage[english]{babel}
\usepackage{graphicx}
\usepackage{caption}
\usepackage[parfill]{parskip}
\usepackage{hyperref}
\usepackage{booktabs}
%\usepackage{setspace}\doublespacing\pagestyle{plain}
\def\inputGnumericTable{}
\usepackage{color}                                            %%
    \usepackage{array}                                            %%
    \usepackage{longtable}                                        %%
    \usepackage{calc}                                             %%
    \usepackage{multirow}                                         %%
    \usepackage{hhline}                                           %%
    \usepackage{ifthen}
\usepackage{array}
\usepackage{amsmath}   % for having text in math mode
\usepackage{parallel,enumitem}
\usepackage{listings}
\lstset{
language=tex,
frame=single, 
breaklines=true
}
  
%Following 2 lines were added to remove the blank page at the beginning
\usepackage{atbegshi}% http://ctan.org/pkg/atbegshi
\AtBeginDocument{\AtBeginShipoutNext{\AtBeginShipoutDiscard}}
%
%New macro definitions
\newcommand{\mydet}[1]{\ensuremath{\begin{vmatrix}#1\end{vmatrix}}}
\providecommand{\brak}[1]{\ensuremath{\left(#1\right)}}
\providecommand{\norm}[1]{\left\lVert#1\right\rVert}
\newcommand{\solution}{\noindent \textbf{Solution: }}
\newcommand{\myvec}[1]{\ensuremath{\begin{pmatrix}#1\end{pmatrix}}}
\let\vec\mathbf
\begin{document}
\begin{center}
\title{\textbf{Parallel Lines}}
\date{\vspace{-5ex}} %Not to print date automatically
\maketitle
\end{center}
\setcounter{page}{1}
\section*{11$^{th}$ Maths - Chapter 10}
This is Problem-6 from Exercise 10.3
\begin{enumerate}
	\item Find the distance between parallel lines 
	
(i) 15x+8y-34=0 and  15x+8y+31=0 \\
(ii) l(x+y)+p=0 and  l(x+y)-r=0
\	
\item solution for problem 1
\\
Given line is 
\begin{align}
	15x+8y-34=0\text{ and }15x+8y+31=0
\end{align}
this equation can be expressed as 
\begin{align}
	\vec{n}^{\top}\vec{x}=c	
\end{align}
\begin{align}
\text{ where }
		\vec{n} = \myvec{15 \\8}&=-34\\
		\vec{n} = \myvec{15 \\8}&= 31
\end{align} 
distance between parallel lines 
\begin{align}
\vec{d}&=\frac{\left|c1-c2\right|}{\norm{n}}\\
\vec{d}&=\frac{\left|-34-31)\right|}{\sqrt{289}}
\end{align}
		
\begin{align}
	\vec{d}=\frac{65}{17}
\end{align}
\begin{figure}[h!]
\begin{center}
\includegraphics[width=\columnwidth]{para.png}
\end{center}
\caption{}
\label{fig:Fig1}
\end{figure}
	\item solution for problem 2
	\\
Given line is 
\begin{align}
l(x+y)+p=0\text{ and }l(x+y)-r=0
\end{align}
this equation can be expressed as 
\begin{align}
	\vec{n}^{\top}\vec{x}=c
\end{align}
\begin{align}
\text{ where }
		\vec{n} = \myvec{l\\l}&=p\\
		\vec{n} = \myvec{l\\l}&=-r
\end{align}
distance between parallel lines 
\begin{align}
\vec{d}&=\frac{\left|p-(-r))\right|}{\sqrt{2l^{2}}}\\
\vec{d}&=\frac{\left|p+r)\right|}{l\sqrt{2}}
\end{align}
		
\begin{figure}[h!]
\begin{center}
\includegraphics[width=\columnwidth]{para1.png}
\end{center}
\caption{}
\label{fig:Fig2}
\end{figure}
\end{enumerate}
\end{document}