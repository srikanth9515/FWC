\documentclass[12pt]{article}
\usepackage{graphicx}
%\documentclass[journal,12pt,twocolumn]{IEEEtran}
\def\inputGnumericTable{}
\usepackage{color}                                            %%
    \usepackage{array}                                            %%
    \usepackage{longtable}                                        %%
    \usepackage{calc}                                             %%
    \usepackage{multirow}                                         %%
    \usepackage{hhline}                                           %%
    \usepackage{ifthen}
\usepackage[none]{hyphenat}
\usepackage{graphicx}
\usepackage{listings}
\usepackage[english]{babel}
\usepackage{graphicx}
\usepackage{caption} 
\usepackage{hyperref}
\usepackage{booktabs}
\usepackage{array}
\usepackage{amsmath}   % for having text in math mode
\usepackage{listings}
\lstset{
  frame=single,
  breaklines=true
}
  
%Following 2 lines were added to remove the blank page at the beginning
\usepackage{atbegshi}% http://ctan.org/pkg/atbegshi
\AtBeginDocument{\AtBeginShipoutNext{\AtBeginShipoutDiscard}}
%


%New macro definitions
\newcommand{\mydet}[1]{\ensuremath{\begin{vmatrix}#1\end{vmatrix}}}
\providecommand{\brak}[1]{\ensuremath{\left(#1\right)}}
\providecommand{\norm}[1]{\left\lVert#1\right\rVert}
\newcommand{\solution}{\noindent \textbf{Solution: }}
\newcommand{\myvec}[1]{\ensuremath{\begin{pmatrix}#1\end{pmatrix}}}
\let\vec\mathbf

\begin{document}

\begin{center}
\title{\textbf{Properties of Parallelegram}}
\date{\vspace{-5ex}} %Not to print date automatically
\maketitle
\end{center}

\setcounter{page}{1}

\section{10$^{th}$ Maths - Chapter 7}

This is Problem-6 from Exercise 7.2

\begin{enumerate}
\item If $\vec{A}(1, 2),\vec{B}(4, x),\vec{C}(y, 6) \text{and } \vec{D}(3, 5)$ are the vertices of a parallelogram taken in order,find x and y.
\end{enumerate}

\solution The input parameters for this problem are available in
\begin{center}
\begin{table}[ht!]
	\input{/home/srikanth/para/tables/table.tex}
\caption{}
\label{table}	
\end{table}
\end{center}
\begin{align}
  \label{eq:det2f}
  \vec{U} &=\brak{\vec{B}-\vec{A}} = \brak{\myvec{4 \\y } - \myvec{1 \\2 } } = \myvec{3 \\y-2 }
 \end{align}
 \begin{align}
   \vec{V} &= \brak{\vec{C}-\vec{D}} = \brak{\myvec{x \\6 } - \myvec{3 \\5 } } = \myvec{x-3 \\1}   
 \end{align}
  
 \begin{align} 
 \myvec{3\\y-2}=\myvec{x-3\\1}
 \end{align}
 \begin{align} 
   x=6 \\ y=3
 \end{align}
 \begin{align}
 \vec{P} &= \brak{\vec{B}-\vec{A}} = \brak{\myvec{4 \\3 } + \myvec{ 1 \\2} } = \myvec{3\\1}  
 \end{align}
 \begin{align}
 \vec{Q} &= \brak{\vec{C}-\vec{D}} = \brak{\myvec{6 \\6 } + \myvec{ 3 \\5} } = \myvec{3\\1}
 \end{align}
 \begin{align}
 \vec{R} &= \brak{\vec{C}-\vec{B}} = \brak{\myvec{6 \\6 } + \myvec{ 4 \\ 3} } = \myvec{2\\3} 
 \end{align}
 \begin{align}
 \vec{S} &= \brak{\vec{D}-\vec{A}} = \brak{\myvec{3 \\5 } + \myvec{1 \\2} } = \myvec{2\\3} 
\end{align}
\begin{figure}[h!]
	\begin{center}
  \includegraphics[width=\columnwidth]{/home/srikanth/para/fig/para.pdf}
	\end{center}
\caption{}
\label{fig:Fig3}
\end{figure}
We know that P=Q and R=S , ABCD is a parallelogram. 


\end{document}
