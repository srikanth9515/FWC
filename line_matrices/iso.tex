\documentclass{article}
\usepackage{amsmath,amssymb,amsfonts,amsthm}
\usepackage{enumitem}
\usepackage{graphicx}
\graphicspath{{./figs/}}{}
\usepackage{tabularx}
\usepackage{lipsum}
\usepackage{xcolor}
\usepackage{listings}
\usepackage{float}
\usepackage{titlesec}
\usepackage{listings}
\usepackage{watermark}
\usepackage{hyperref,xcolor}
\usepackage[e]{esvect}
%\newcommand{\myvec}[2][1mu]{\vec{#2\mkern-#1}\mkern#1}
\newcommand{\myvec}[1]{\ensuremath{\begin{pmatrix}#1\end{pmatrix}}}
\hypersetup{
    colorlinks,
    urlcolor={black}	%black!50!blue
}
\providecommand{\mtx}[1]{\mathbf{#1}}
\newcommand{\Mod}[1]{\ (\mathrm{mod}\ #1)}
\let\vec\mathbf

\def\inputGnumericTable{}
\usepackage{array}
\usepackage{longtable}
\usepackage{calc}
\usepackage{multirow}
\usepackage{hhline}
\usepackage{ifthen}

\providecommand{\cbrak}[1]{\ensuremath{\left\{#1\right\}}}
\newcommand{\Problem}{\noindent \textbf{Problem: }}
\newcommand{\solution}{\noindent \textbf{Solution: }}
\setlist[enumerate]{font=\small\bfseries}
\newcommand{\figuremacro}[5]{
    
}
\lstset{
frame=single, 
breaklines=true,
columns=fullflexible
}
\begin{document}

\title{Assignment: Isosceles triangle}
\author{\Large S.Srikanth Reddy - FWC220107}
\date{}
\maketitle
\Problem Check whether$(5,-2),(6,4) \text{ and } (7,-2)$ are the vertices of an isosceles triangle.
\begin{table}[h!]
	\small
	\centering
	%\subimport{../tables/}{table1.tex}
     \input{/home/srikanth/vectore/tables/table1.tex}
	%\caption{}
	\label{table:12table1}
	\end{table}

\solution 1:
Let the given points be $\vec{A}, \vec{B}, \vec{C}$ respectively. 
Then, the direction vectors of $AB, BC$ and $CA$ are
\begin{align}
\vec{A}-\vec{B}&=\myvec{5\\-2}-\myvec{6\\4}=\myvec{-1\\-6}\\
\vec{B}-\vec{C}&=\myvec{6\\4}-\myvec{7\\-2}=\myvec{-1\\6}\\
\vec{C}-\vec{A}&=\myvec{7\\-2}-\myvec{5\\-2}=\myvec{2\\0}\\
{\vec{A} -\vec{B}}^\brak{\top}{\vec{B} -\vec{C}}&=  \myvec{-1 & -6}\myvec{-1 \\ 6}\\
&=37\\
{\vec{B} -\vec{C}}^\brak{\top}{\vec{C} -\vec{A}}&=  \myvec{-1 & 6}\myvec{2 \\ 0}\\
&=-2\\
{\vec{C} -\vec{A}}^\brak{\top}{\vec{A} -\vec{B}}&=  \myvec{2 & 0}\myvec{-1 \\ -6}\\
&=-2\\
From  the above equations,
{\vec{A} -\vec{B}}\perp {\vec{B} -\vec{C}}\\
\angle BCA = \angle CAB  
\end{align}
Thus, the triangle is isosceles triangle.

\textbf{Method2}: Let the given points be $\vec{E}, \vec{F}, \vec{D}$ respectively. 
 Then, the direction vectors of  $BD ,CF,AE$ are
\begin{align}
\vec{B}-\vec{C}&= \myvec{6 \\ 4}-\myvec{7 \\-2} = \myvec{13/2 \\ 1}\\
\vec{C}-\vec{A}&= \myvec{7 \\-2}-\myvec{5 \\-2} = \myvec{6 \\ -2}\\
\vec{A}-\vec{B}&= \myvec{5 \\ -2} -\myvec{6 \\4} = \myvec{11/2 \\ 1}\\
%From the above,  we find that midpoints values .\\
{\vec{E}-\vec{A}}^\brak{\top}{\vec{C}-\vec{B}}&= \myvec{3/2 & 3}\myvec{-1 \\ -6}\\
&\neq0\\
{\vec{F}-\vec{C}}^{\brak\top}{\vec{B}-\vec{A}}&= \myvec{3/2 & 3}\myvec{1 \\ 6}\\
&\neq 0\\
{\vec{D}-\vec{B}}^\brak{\top}{\vec{C}-\vec{A}}&= \myvec{0 & -6}\myvec{2 \\ 0}\\
&=0 \\
{\vec{B} -\vec{D}}\perp {\vec{A} -\vec{C}}\\
\angle BCA = \angle CAB  
\end{align}

\pagebreak\section{Figure}
\begin{figure}[h]
\includegraphics[width=\columnwidth]{par.pdf}
\caption{Isosceles triangle}
		\label{fig:Figure}
\end{figure}
\begin{lstlisting}
https://github.com/srikanth9515/FWC/tree/main/maths/vec1
\end{lstlisting}
\end{document}
